\documentclass[12pt]{article}
\usepackage{inputenc}
\usepackage[greek,english]{babel}
\usepackage{alphabeta}
\usepackage{graphicx}
\usepackage{tabularx}
\usepackage{float}
\usepackage{multirow}
\usepackage{rotating}
\usepackage{diagbox}
\usepackage[table]{xcolor}
\usepackage{makecell}
\addto\captionsenglish{
	\renewcommand\tablename{Πινακας}
}
\usepackage{cite}
\usepackage{amsmath}
\usepackage{amssymb}
\author{Βασίλειος Ροδανίδης 12164}
\title{Ασκήσεις Φυλλου εργασιας 1}
\date{}
\begin{document}
\maketitle
\section{Άσκηση 1}
$(i) \sqrt{25{n^2} + 3n} - 2n \in \Theta(n)$ $$ $$
$$ $$
Θετουμε $t(n)= \sqrt{25{n^2} + 3n} - 2n$ και $g(n)= n$ $$ $$
$$ $$ Παιρνουμε το οριο $\lim_{n \to \infty} \dfrac{t(n)}{g(n)}$ $$ $$
$\lim_{n \to \infty} \dfrac{t(n)}{g(n)}=\lim_{n \to \infty} \dfrac{\sqrt{25{n^2} + 3n} - 2n}{n}=\lim_{n \to \infty} \dfrac{\sqrt{{n^2}(25 + 3/n)} - 2n}{n}=\lim_{n \to \infty} \dfrac{n \sqrt{(25 + 3/n)} - 2n}{n}=\lim_{n \to \infty} \dfrac{1  \sqrt{(25 + 3/n)} - 2 \cdot 1}{1}=$ $$ $$ $=\lim_{n \to \infty} \sqrt{(25 + 3/n)} - 2=\lim_{n \to \infty} \sqrt{25} - 2=\lim_{n \to \infty} 5-2=\lim_{n \to \infty} 3=$ $$ $$ $ \implies\lim_{n \to \infty} \dfrac{t(n)}{g(n)}=3=c>0$ $$ $$
Αρα $t(n) \in \Theta (g(n))$ $$ $$
Αρα $\sqrt{25{n^2} + 3n} - 2n \in \Theta (n)$


$$ $$ $$ $$
(ii) $\log n$ $\in  \Omicron(2^{\sqrt{log n}} )$
$$ $$
Θετουμε $t(n)= \log n$ και $g(n)= 2^{\sqrt{log n}}$ $$ $$
$$ $$ Παιρνουμε το οριο $\lim_{n \to \infty} \dfrac{t(n)}{g(n)}$ $$ $$
$\lim_{n \to \infty} \dfrac{t(n)}{g(n)}=\lim_{n \to \infty} \dfrac{\log n}{2^{\sqrt{\log n}}}$ $$ $$
θετω $w=\log n$ ,$ n \to \infty \implies \log n \to \infty \implies w \to \infty$ $$ $$
$\implies\lim_{n \to \infty} \dfrac{\log n}{2^{\sqrt{\log n}}}=$ 
$\lim_{w \to \infty} \dfrac{w}{2^{\sqrt{w}}}  \overset{\dfrac{\infty}{\infty}} {\implies}\lim_{w \to \infty} \dfrac{1}{(2^{\sqrt{w}})'}=...=\lim_{w \to \infty} \dfrac{c}{\infty}=$ $$ $$
$=0$ $$ $$
$ \implies\lim_{n \to \infty} \dfrac{t(n)}{g(n)}=0=c$ $$ $$
Αρα $t(n) \in \Omicron (g(n))$ $$ $$
Αρα $logn \in \Omicron (2^{\sqrt{log n}})$ $$ $$ $$ $$

(iii) $log(n!) \in \Theta(nlogn)$
$$ $$
Θετουμε $t(n)= \log (n!)$ και $g(n)= nlogn $ $$ $$
$$ $$ Παιρνουμε το οριο $\lim_{n \to \infty} \dfrac{t(n)}{g(n)}$ $$ $$
$\lim_{n \to \infty} \dfrac{t(n)}{g(n)}=\lim_{n \to \infty} \dfrac{\log (n!)}{nlogn}= \lim_{n \to \infty} \dfrac{\log (n!)}{log(n^n)}$ $$ $$
Ισχυει οτι: $ n^n=\underbrace{\cdot n \cdot n \cdot... \cdot n}_{n\text{ φορες}}> n \cdot (n-1) \cdot (n-2) \cdot ... \cdot 1= n!$ $$ $$
$\implies n^n > n! \overset{logx \text{ αυξουσα}} {\implies} log(n^n)>log(n!) $ Αρα στο κλασμα μας ο παρονομαστης μηδενιζεται με γρηγοροτερο ρυθμο απ οτι ο αριθμητης επομενως :$$ $$
$\implies \lim_{n \to \infty} \dfrac{\log (n!)}{log(n^n)}=\lim_{n \to \infty} \dfrac{1}{\infty}=0$ $$ $$
Αρα $t(n) \in \Theta (g(n)) \implies t(n) \in \Omicron (g(n)) $ $$ $$
Αρα $log(n!) \in \Omicron (nlogn)$ $$ $$ $$ $$



$$ $$
\section{Ασκηση 2}
(i) $SUM_XSYGKRISI(A[1,...,n])$$$ $$
$    akeraios_x=x $ $$ $$
    for (i=0,...,n; i++) do $$ $$
    $\qquad$	for(j=(i+1),...,n-1; j++) do$$ $$
    $\qquad$ $\qquad$	if(A[i]+A[j]==x) $$ $$
    $\qquad$ $\qquad$ $\qquad$return true; $$ $$
    $\qquad$ $\qquad$ $\qquad$	else$$ $$
   $\qquad$ $\qquad$ $\qquad$    return false;$$ $$ $$ $$
    	
    	Το συνολικο πληθος προσθεσης ενος ζευγους αριθμων και συγκρισης του αθροισματος με τον ακεραιο αριθμο x ειναι:$$ $$
    	$1+2+3+...+(n-1)=\frac{1}{2} n (n-1)=O(n^2)$ $$ $$ $$  $$
    	
    	
(ii)Για το δευτερο προβλημα θα χωρισουμε τον πινακα Α[1,...,n] σε δυο υποπινακες Α[1,...,$\frac{n}{2}$] και Α[$\frac{n}{2}+1$,...n].Ετσι θα κανουμε merge 2 ταξινομημενων πινακων μεγεθους $\frac{n}{2}$ και $\frac{n}{2}$ πραγμα το οποιο γινεται με 2$\frac{n}{2}$ - 1=n-1 συγκρισεις.$$ $$

π.χ. Α[1,3,7,8,2,4,5,11]
Mε την Merge σπαει ο πινακας σε Α[1,3,7,8] και A[2,4,5,11]$$ $$
και στην συνεχεια ακολουθειται η εξης διαδικασια:$$ $$
Προσθετω 1 με 2 και συγκρινω με τον x,$$ $$
Προσθετω 1 με 4 και συγκρινω με τον x,$$ $$
Προσθετω 1 με 5 και συγκρινω με τον x,$$ $$ 
Προσθετω 1 με 11 και συγκρινω με τον x,$$ $$
Προσθετω 3 με 2 και συγκρινω με τον x,$$ $$
Προσθετω 3 με 4 και συγκρινω με τον x,$$ $$
Προσθετω 3 με 5 και συγκρινω με τον x,$$ $$
Προσθετω 3 με 11 και συγκρινω με τον x,$$ $$
Προσθετω 7 με 2 και συγκρινω με τον x,$$ $$
Προσθετω 7 με 4 και συγκρινω με τον x,$$ $$
Προσθετω 7 με 5 και συγκρινω με τον x,$$$$
Προσθετω 7 με 11 και συγκρινω με τον x,$$$$
Προσθετω 8 με 2 και συγκρινω με τον x,$$$$
Προσθετω 8 με 4 και συγκρινω με τον x,$$$$
Προσθετω 8 με 5 και συγκρινω με τον x,$$$$
Προσθετω 8 με 11 και συγκρινω με τον x.$$$$ $$$$

Οπου για την Mergesort γνωριζουμε  $\to$ Ο(nlogn).

\section{Άσκηση 3}
ΜΕΘΟΔΟΣ 1$$ $$
Ορίζω εναν πινακα (πλαγια στοιβα) που υποστηριζει τις παρακατω πραξεις $$ $$
$1. $ Push(A[i],x):Ενθετει το στοιχειο x στην αριστεροτερη ελευθερη θεση του πινακα Α[i] με κοστος: Ο(i)σταθερο $$ $$
$2. $ Pop(A[i]):Επιστρεφει και διαγραφει το στοιχειο που μπηκε τελευταιο στον πινακα Α[i] με κοστος : Ο(i) σταθερο $$ $$
$3. $ Multipop(A[i],m):Επιστρεφει και διαγραφει απ την Α[i] τα τελευταια $ min\{m,|A[i]|\} $στοιχεια με γραμμικο κοστος ως προς το $min\{m,|A[i]|\}: o(min\{m,|A[i]|\}$ $$ $$
$$ $$
Ο πινακας Α[i] θα εχει την εξης μορφη βλ. Πινακα 1
\begin{table}[H]
\begin{center}
\caption{Αυτος ειναι ο Πινακας A[i]}\label{tbl:1}
\begin{tabular}{|l|c|r|c|l|}
\hline
\ $*$ & $*$ & $*$ & $...$ & $  $\\
\hline
\end{tabular}
\end{center}
\end{table}
$$ $$ 

Ο πινακας ειναι συμπληρωμενος με k στοιχεια (οπου * τυχαιο στοιχειο).
Αν εφαρμοσω την push τοτε:$$ $$
1. Εαν k=i δηλαδη ο πινακας ειναι πληρης τοτε παιρνω εναν καινουργιο πινακα $A[2i]$ με k+1 στοιχεια  της μορφης βλ. Πινακα2$$ $$

$$ $$
\begin{table}[H]
\begin{center}
\caption{Πινακας $A[2i]$}
\begin{tabular}{|l|c|r|c|l|l|c|r|}
\hline
\ $*$ & $*$ & $*$ & $...$ & $  $ & $  $ & $  $\\
\hline
\end{tabular}
\end{center}
\end{table}
2. Εαν k $<$i τοτε παιρνω τον ιδιο πινακα $A[i]$ με k+1 στοιχεια ο οποιος θα ειναι της μορφης βλ. Πινακα 3
\begin{table}[H]
\begin{center}
\caption{Πινακας $A[i]$}
\begin{tabular}{|l|c|r|c|l|l|c|r|}
\hline
\ $*$ & $*$ & $*$ & $*$ & $...$ & $ $\\
\hline
\end{tabular}
\end{center}
\end{table}       
$$ $$ $$ $$ 
Αν εφαρμοσω την pop τοτε:$$ $$
Παιρνω τον ιδιο πινακα $A[i]$ με k-1 στοιχεια ο οποιος θα ειναι της μορφης βλ. Πινακα 4
$$ $$
\begin{table}[H]
\begin{center}
\caption{Πινακας $A[i]$}
\begin{tabular}{|l|c|r|c|l|l|c|r|}
\hline
\ $*$ & $*$ & $...$ & $  $ & $  $ & $  $\\
\hline
\end{tabular}
\end{center}
\end{table}
Αν εφαρμοσω την Multipop τοτε:$$ $$
Παιρνω τον ιδιο πινακα $A[i]$ με max\{0,k-$min\{m,|A[i]|\}$ στοιχεια ο οποιος θα ειναι της μορφης βλ. Πινακα 5
$$ $$
\begin{table}[H]
\begin{center}
\caption{Πινακας $A[i]$}
\begin{tabular}{|l|c|r|c|l|l|c|r|}
\hline
\ $*$ & $...$ & $ $ & $  $ & $  $ & $  $\\
\hline
\end{tabular}
\end{center}
\end{table}
$$ $$
 (i)Αναλυση χειροτερης περιπτωσης$$ $$
 n πραξεις, ολες ειναι ακριβες $\implies$ ολες εχουν γραμμικο κοστος$$ $$
 Συνολικο κοστος Ο($n^2$) τετραγωνικο
 (ii) Επιμερισμενο κοστος $$ $$ $$ $$ 
 
 ΜΕΘΟΔΟΣ 2$$ $$
 Ορίζω εναν πινακα (πλαγια στοιβα) που υποστηριζει τις παρακατω πραξεις $$ $$
$1. $ Push(A[i],x):Ενθετει το στοιχειο x στην αριστεροτερη ελευθερη θεση του πινακα Α[i] με κοστος: 1 $$ $$
$2. $ Pop(A[i]):Επιστρεφει και διαγραφει το στοιχειο που μπηκε τελευταιο στον πινακα Α[i] με κοστος : 1$$ $$
$3. $ Multipop(A[i],m):Επιστρεφει και διαγραφει απ την Α[i] τα τελευταια $ min\{m,|A[i]|\} $στοιχεια με γραμμικο κοστος ως προς το $min\{m,|A[i]|\}: (min\{m,|A[i]|\}$ $$ $$
$$ $$
Ο πινακας Α[i] θα εχει την εξης μορφη βλ. Πινακα 6
\begin{table}[H]
\begin{center}
\caption{Αυτος ειναι ο Πινακας A[i]}\label{tbl:1}
\begin{tabular}{|l|c|r|c|l|}
\hline
\ $*$ & $*$ & $*$ & $...$ & $  $\\
\hline
\end{tabular}
\end{center}
\end{table}
$$ $$

Ο πινακας ειναι συμπληρωμενος με k στοιχεια (οπου * τυχαιο στοιχειο).
Αν εφαρμοσω την push τοτε:$$ $$
1. Εαν k=i τοτε παιρνω εναν καινουργιο πινακα $A[2i]$ με k+1 στοιχεια  της μορφης βλ. Πινακα7$$ $$

$$ $$
\begin{table}[H]
\begin{center}
\caption{Πινακας $A[2i]$}
\begin{tabular}{|l|c|r|c|l|l|c|r|}
\hline
\ $*$ & $*$ & $*$ & $...$ & $  $ & $  $ & $  $\\
\hline
\end{tabular}
\end{center}
\end{table}
2. Εαν k $<$i τοτε παιρνω τον ιδιο πινακα $A[i]$ με k+1 στοιχεια ο οποιος θα ειναι της μορφης βλ. Πινακα 8
\begin{table}[H]
\begin{center}
\caption{Πινακας $A[i]$}
\begin{tabular}{|l|c|r|c|l|l|c|r|}
\hline
\ $*$ & $*$ & $*$ & $*$ & $...$ & $ $\\
\hline
\end{tabular}
\end{center}
\end{table}       
$$ $$ $$ $$ 
Αν εφαρμοσω την pop τοτε:$$ $$
Παιρνω τον ιδιο πινακα $A[i]$ με k-1 στοιχεια ο οποιος θα ειναι της μορφης βλ. Πινακα 9
$$ $$
\begin{table}[H]
\begin{center}
\caption{Πινακας $A[i]$}
\begin{tabular}{|l|c|r|c|l|l|c|r|}
\hline
\ $*$ & $*$ & $...$ & $  $ & $  $ & $  $\\
\hline
\end{tabular}
\end{center}
\end{table}
Αν εφαρμοσω την Multipop τοτε:$$ $$
Παιρνω τον ιδιο πινακα $A[i]$ με max\{0,k-$min\{m,|A[i]|\}$ στοιχεια ο οποιος θα ειναι της μορφης βλ. Πινακα 10
$$ $$
\begin{table}[H]
\begin{center}
\caption{Πινακας $A[i]$}
\begin{tabular}{|l|c|r|c|l|l|c|r|}
\hline
\ $*$ & $...$ & $ $ & $  $ & $  $ & $  $\\
\hline
\end{tabular}
\end{center}
\end{table}
$$ $$
 (i)Αναλυση χειροτερης περιπτωσης$$ $$
 Εχω n πραξεις, το πολυ εχω n push αρα το κοστος χειριστης περιπτωσης ειναι 2n. Για Pop/Multipop δεν εχω καποιο κοστος.
 Συνολικο κοστος n πραξεων $\leq 2n$ $\implies$ Επιμερισμενο κοστος $\leq \frac{2n}{n}=2$ $$ $$
 (ii) Για τον δευτερο πινακα εχουμε την διαφορα οτι o πινακας $A[i]$ στην εκτελεση της pop εχει την παρακατω ιδιαιτεροτητα:
 
Για k-1$>i/2$ παιρνω τον ιδιο πινακα $A[i]$ με k-1 στοιχεια ο οποιος θα ειναι της μορφης βλ. Πινακα 11
$$ $$
\begin{table}[H]
\begin{center}
\caption{Πινακας $A[i]$}
\begin{tabular}{|l|c|r|c|l|l|c|r|}
\hline
\ $*$ & $*$ & $...$ & $  $ & $  $ & $  $\\
\hline
\end{tabular}
\end{center}
\end{table}
Για k-1$\leq i/2$ παιρνω εναν καινουργιο πινακα $A[i/2]$ με k-1 στοιχεια ο οποιος θα ειναι της μορφης βλ. Πινακα 12
$$ $$
\begin{table}[H]
\begin{center}
\caption{Πινακας $A[i]$}
\begin{tabular}{|l|c|r|c|l|l|c|r|}
\hline
\ $*$ & $...$ & $  $\\
\hline
\end{tabular}
\end{center}
\end{table}$$ $$



\section{Ασκηση 4}
(i)Θα αναπαραστησουμε καθε αλγοριθμο ταξινομησης με συγκρισεις στο δυαδικο δεντρο:$$ $$
$$ $$ 
\begin{center}
A[1] $\leq A[2] $ $$ $$
Αν ισχυει $\downarrow$ $\qquad$ Αν δεν ισχυει $\downarrow$ $$ $$
A[2] $\leq$ A[3] $\qquad$ A[1] $\leq$ A[3]$$ $$
Αν ισχυει $\downarrow$ $\qquad$ Αν δεν ισχυει $\downarrow$ $\qquad$ Αν ισχυει $\downarrow$ $\qquad$ Αν δεν ισχυει $\downarrow$ $$ $$
A[1]$\leq$A[2]$\leq$A[3]$\qquad$ A[1]$\leq$A[3]$\qquad$ A[2]$\leq$A[1]$\leq$A[3]$\qquad$ A[2]$\leq$A[3]$$ $$ 
------$\qquad$ $\qquad$ $\qquad$ $\Downarrow$ $\qquad$ $\qquad$ $\qquad$ ------ $\qquad$ $\qquad$ $\qquad$ $\Downarrow$ $$ $$
Αν ισχυει $\downarrow$  Αν δεν ισχυει $\downarrow$ $\qquad$ Αν ισχυει $\downarrow$ Αν δεν ισχυει $\downarrow$ $$ $$
A[1]$\leq$A[3]$\leq$A[2] A[3]$\leq$A[1]$\leq$A[2]$\qquad$ A[2]$\leq$A[3]$\leq$A[1] A[3]$\leq$A[2]$\leq$A[1]$$ $$ $$ $$ $$ $$
\end{center}
1. Τα φυλλα του δεντρου Τ ειναι τουλαχιστον n!$$ $$
2. Ο αλγοριθμος εκτελει h(Τ] συγκρισεις στην χειροτερη περιπτωση, οπoυ h(T) ειναι το υψος του Τ.$$ $$
3. Το Τ εχει n! φυλλα $\implies$ h(T)$\geq log_2 (n!)=log(1 \cdot 2 \cdot ... \cdot \frac{n}{2} \cdot ...\cdot n)\geq log(\frac{n}{2}...n)\geq log(\frac{n}{2}...\frac{n}{2})=log(\frac{n}{2})^\frac{n}{2}=\frac{n}{2} \cdot log\frac{n}{2}=O(nlogn)$ $$ $$
Aρα απο 1,2,3 $\implies$ το πληθος συγκρισεων καθε αλγοριθμου ταξινομησης ειναι $\Omega(nlogn)$ $$ $$ $$ $$
(ii)
1. Τα φυλλα του δεντρου Τ2 ειναι τουλαχιστον 2n$$ $$
2. Ο αλγοριθμος εκτελει h(Τ2] συγκρισεις στην χειροτερη περιπτωση, οπoυ h(T2) ειναι το υψος του Τ2.$$ $$
3. Το Τ2 εχει 2n φυλλα $\implies$ h(T2)$\geq log_2 (2n)=log_2 2 + log_2 n= 1 + log_2 n\geq log_2 n= O(logn)$ $$ $$
Aρα απο 1,2,3 $\implies$ το προβλημα της αποφασης ειναι της ταξης του $\Omega(logn)$ $$ $$ $$ $$

Το δενδρο θα ειναι της μορφης:$$ $$

πχ x ενας ακεραιος και A[i] i=1,..,4 ενα ταξινομημενο διανυσμα με 4 στοιχεια$$ $$
\begin{center}
x=A[1] $$ $$
Αν ισχυει $\downarrow$ $\qquad$ Αν δεν ισχυει $\downarrow$ $$ $$
-----------$\qquad$ x=A[2]$$ $$
$\qquad$ $\qquad$ $\qquad$Αν ισχυει $\downarrow$ $\qquad$ Αν δεν ισχυει $\downarrow$ $$ $$
$\qquad$ $\qquad$ $\qquad$-----------$\qquad$ x=A[3]$$ $$
$\qquad$ $\qquad$ $\qquad$ $\qquad$ $\qquad$ $\qquad$Αν ισχυει $\downarrow$ $\qquad$ Αν δεν ισχυει $\downarrow$ $$ $$
$\qquad$ $\qquad$ $\qquad$ $\qquad$ $\qquad$ $\qquad$ -----------$\qquad$ x=A[4]$$ $$
\end{center}







\end{document}